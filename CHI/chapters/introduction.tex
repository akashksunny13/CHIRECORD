\documentclass{article}
\usepackage{amsmath}
\begin{document}
	
	\chapter{\huge AIM\\}
	
	Familiarization of the components / Cards inside a computer, standard connectors, cords, different ports,
		and various computer peripherals\\
	
  	\chapter{\large\textbf{\\COMPUTER COMPONENTS}}
  		
	\section{MOTHERBOARD}
	The motherboard is the main component inside the case. It is a large rectangular board with integrated circuitry
	that connects the other parts of the computer including the CPU, the RAM, the disk drives as well as any peripherals connected via the ports or the expansion slots.
	\section{PROCESSOR}
	A central processing unit (CPU), also referred to as a central processor unit,
	[1] is the hardware within
	a computer that carries out the instructions of a computer program by performing the basic arithmetical, logical,
	and input/output operations of the system. A computer can have more than one CPU; this is
	called multiprocessing.Two typical components of a CPU are the arithmetic logic unit (ALU), which
	performs arithmetic and logical operations, and the control unit (CU), which extracts instructions
	from memory and decodes and executes them, calling on the ALU when necessary.
\section{CHIPSET}
A chipset is a set of electronic components in an integrated circuit that manage the data flow between the
processor, memory and peripherals. It is usually found in the motherboard of a computer. Because it controls communications between the
processor and external devices, the chipset plays a crucial role in determining system performance. Based
on Intel Pentium-class microprocessors, the term chipset often refers to a specific pair of chips on the
motherboard: the northbridge and the southbridge. The northbridge links the CPU to very high-speed devices,
especially RAM and graphics controllers, and the southbridge connects to lower-speed peripheral buses (such
as PCI or ISA). 
\section{READ ONLY MEMORY}
Read-only memory (ROM) is a class of storage medium used in computers and other electronic devices. Data
stored in ROM cannot be modified, or can be modified only slowly or with difficulty, so it is mainly used to
distribute firmware (software that is very closely tied to specific hardware and unlikely to need frequent
updates). Other types of non-volatile memory such as erasable programmable read only memory (EPROM)
and electrically erasable programmable read-only memory (EEPROM or Flash ROM) are sometimes referred to,
in an abbreviated way, as "read-only memory" (ROM); although these types of memory can be erased and reprogrammed
multiple times, writing to this memory takes longer and may require different procedures than
reading the memory.
\section{BIOS}
In IBM PC compatible computers, the Basic Input/Output System (BIOS), also known as the system
BIOS or ROM BIOS, is a de facto standard defining a firmware interface. The fundamental purposes of the
BIOS are to initialize and test the system hardware components, and to load a bootloader or an operating
system from a mass memory device. The BIOS additionally provides abstraction layer for the hardware, i.e. a
consistent way for application programs and operating systems to interact with the keyboard, display, and other
input/output devices. Variations in the system hardware are hidden by the BIOS from programs that use BIOS
services instead of directly accessing the hardware.
\section{BUSES}
Buses connect the CPU to various internal components and to expansion cards for graphics and sound. It is a
physical arrangement that provides the same logical functionality as a parallel electrical bus.\\
The various Bus architectures currently include:
\subsection{PCI Express}
 PCI Express (Peripheral Component Interconnect Express), officially abbreviated as PCIe, is
 a high-speed serial computer expansion bus standard designed to replace the older PCI, PCI-X,
 and AGP bus standards. PCIe has higher maximum system bus throughput, lower I/O pin count and smaller physical footprint,
 better performance-scaling for bus devices, a more detailed error detection and reporting mechanism
 (Advanced Error Reporting (AER)), and native hot-plug functionality.
 \subsection{PCI}
  PCI, is a local computer bus for attaching hardware devices in a computer. The PCI bus supports the
  functions found on a processor bus, but in a standardized format that is independent of any particular
  processor. Devices connected to the bus appear to the processor to be connected directly to the processor
  bus, and are assigned addresses in the processor's address space.. Typical PCI cards used in PCs include: network cards, sound
  cards, modems, extra ports such as USB or serial, TV tuner cards and disk controllers.
  \subsection{SATA}
   Serial ATA (Advance Technology Attachment)(SATA) is a computer bus interface that
   connects host bus adapters to mass storage devices such as hard disk drives and optical drives. Serial ATA
   replaces the older AT Attachment standard (ATA later referred to as Parallel ATA or PATA), offering
   several advantages over the older interface: reduced cable size and cost (seven conductors instead of 40),
   native hot swapping, faster data transfer through higher signalling rates, and more efficient transfer through
   an (optional) I/O queuing protocol.
   \section{PORTS}
   A port serves as an interface between the computer and other computers or peripheral devices.Ports are classified as either serial ports or parallel ports
   based on the mode of data transfer. Hot-swappable ports can be connected while equipment is running. Plug-and-play ports are designed so that the connected
   devices automatically start handshaking as soon as the hot-swapping is done. USB ports and FireWire ports are
   plug-and-play.The most commonly used ports in a computer are:
   \subsection{SERIAL PORT}
    In computing, a serial port is a serial communication physical interface through which
    information transfers in or out one bit at a time (in contrast to a parallel port).Throughout most of the
    history of personal computers, data was transferred through serial ports to devices such as modems,
    terminals and various peripherals.The common applications for the serial port include
    dial up modems, GPS receivers, Bar code scanners, Serial mouse etc.
\subsection{PARALLEL PORT}
 A parallel port is a type of interface found on computers (personal and otherwise) for
connecting peripherals. In computing, a parallel port is a parallel communication physical interface. It is
also known as a printer port or Centronics port. The parallel port is usually implemented using a 25 pin DB-
25 connector.
\subsection{ USB}
 Universal Serial Bus (USB) is an industry standard developed in the mid-1990s that defines the
cables, connectors and communications protocols used in a bus for connection, communication, and power
supply between computers and electronic devices. USB was designed to standardize the connection
of computer peripherals (including keyboards, pointing devices, digital cameras, printers, portable media
players, disk drives and network adapters) to personal computers, both to communicate and to
supply electric power. USB has effectively replaced a variety of earlier interfaces, such as serial and parallel ports,
as well as separate power chargers for portable devices. The presently used USB standard is USB 3.0 which
has a data rate of upto 5 GB/sec.
\subsection{ SCSI}
 Small Computer System Interface (SCSI) is a set of standards for physically connecting and
transferring data between computers and peripheral devices. The SCSI standards define commands,
protocols and electrical and optical interfaces. SCSI is most commonly used for hard disks and tape drives,
but it can connect a wide range of other devices, including scanners and CD drives, although not all
controllers can handle all devices. SCSI is an intelligent, peripheral, buffered, peer to peer interface. It hides
the complexity of physical format. Every device attaches to the SCSI bus in a similar manner. Up to 8 or 16
devices can be attached to a single bus. There can be any number of hosts and peripheral devices but there
should be at least one host.
\subsection{ ESATA}
 Standardized in 2004, eSATA (e standing for external) provides a variant of SATA meant for
external connectivity. It uses a more robust connector, longer shielded cables, and stricter (but backwardcompatible)
electrical standards.
\subsection{ FIREWIRE}
 Firewire (IEEE 1394) is a serial bus interface standard for high-speed communications
and isochronous real-time data transfer.The system is commonly used to connect data storage
devices and DV (digital video) cameras, but is also popular in industrial systems for machine vision and
professional audio systems. It is preferred over the more common USB for its greater effective speed and
power distribution capabilities. Firewire supports data transfer rates of up to 3200 Mbits/sec.
\section{EXPANSION DEVICES}
The expansion card (also expansion board, adapter card or accessory card) in is a printed circuit board that can
be inserted into an electrical connector, or expansion slot on a computer motherboard, backplane or riser card to
add functionality to computer system via the expansion bus. The primary purpose of an expansion card is to
provide or expand on features not offered by the motherboard. Some commonly used expansion cards are:
\subsection{VIDEO CARD} 
A video card is an expansion card which generates a feed of output images to a display.
Most video cards offer various functions such as accelerated rendering of 3D scenes and 2D graphics,
MPEG-2/MPEG-4 decoding, TV output, or the ability to connect multiple monitors (multi-monitor). It is
also called a video controller or graphics controller.
\subsection{SOUND CARD}
 A sound card (also known as an audio card) is an internal computer expansion card that
facilitates the input and output of audio signals to and from a computer under control of computer
programs. Typical uses of sound cards include providing the audio component for multimedia applications
such as music composition, editing video or audio, presentation, education and entertainment (games) and
video projection.
\subsection{ NETWORK INTERFACE CONTROLLER CARD}
 A network interface controller (NIC) is a computer hardware component
that connects a computer to a computer network. The network controller implements the electronic circuitry
required to communicate using a specific physical layer and data link layer standard such as Ethernet, WiFi
or Token Ring.
\subsection{ TV TUNER CARD}
 A TV tuner card is a kind of television tuner that allows television signals to be received by
a computer. Most TV tuners also function as video capturecards, allowing them to record television
programs onto a hard disk much like the digital video recorder (DVR) does.
\section {SECONDARY STORAGE DEVICES}
Computer data storage, often called storage or memory, is a technology consisting of computer components
and recording media used to retain digital data. It is a core function and fundamental components of
computers. In practice, almost all computers use a storage hierarchy, which puts fast but expensive and volatile
small storage options close to the CPU and slower but larger, permanent and cheaper options farther away. The
permanent storage is usually referred to as secondary storage. Secondary storage devices can be broadly
classified into two:
\subsection {FIXED MEDIA}
\paragraph{ HARD DISK DRIVES}
  Hard drive, hard disk, or disk drive is a device for
storing and retrieving digital information, primarily computer data. It consists of one or more rigid rapidly rotating discs (often referred to as platters), coated with magnetic material and
with magnetic heads arranged to write data to the surfaces and read it from them. An HDD retains its
data even when powered off. Data is read in a random-access manner. HDDs are
connected to systems by standard interface cables such as SATA (Serial ATA), USB or SAS (Serial
attached SCSI) cables. The capacity of modern hard drives ranges from 500 GB to 4 TB.
\paragraph{ SOLID STATE DRIVES}
A solid-state drive (SSD), sometimes called a solid-state disk or electronic disk, is a
data storage device that uses solid-state memory to store persistent data with the intention of providing
access in the same manner of a traditional block I/O hard disk drive. SSDs are distinguished from
traditional magnetic disks such as hard disk drives (HDDs) or floppy disk, which are electromechanical
devices containing spinning disks and movable read/write heads. Compared with electromechanical
disks, SSDs are typically more resistant to physical shock, run more quietly, have lower access time,
and less latency. The capacity of modern
SSDs usually ranges from 64 GB to 1 TB.
\subsection{ REMOVABLE MEDIA}
\paragraph{OPTICAL DISK DRIVES}
 An optical disc drive (ODD) is a disk drive that uses laser light or electromagnetic
waves within or near the visible light spectrum as part of the process of reading or writing data to or
from optical discs. Some drives can only read from discs, but recent drives are commonly both readers
and recorders, also called burners or writers. Compact discs, DVDs, and Blu-ray discs are common
types of optical media which can be read and recorded by such drives.
\paragraph{ FLOPPY DISK DRIVES} 
They are used for reading and writing to floppy disks, an outdated storage media
consisting of a thin disk of a flexible magnetic storage medium. These were once standard on most
computers but are no longer in common use. Floppy disks, initially as 8-inch (200 mm) media and later
in 5.25-inch (133 mm) and 3.5-inch (90 mm) sizes, were a ubiquitous form of data storage and
exchange from the mid-1970s well into the first decade of the 21st century. Floppies are used today
mainly for loading device drivers not included with an operating system release.
\paragraph{USB Flash drives}
 A USB flash drive is a data storage device that includes flash memory with an
integrated Universal Serial Bus (USB) interface. USB flash drives are typically removable and
rewritable, and physically much smaller than an optical disc. Modern USB drives can store data up to
256 GB.
\paragraph{TAPE DRIVES}
 A tape drive is a data storage device that reads and writes data on a magnetic
tape. Magnetic tape data storage is typically used for offline, archival data storage. Tape media
generally has a favorable unit cost and long archival stability. A tape drive provides sequential
access storage, unlike a disk drive, which provides random access storage. A disk drive can move to
any position on the disk in a few milliseconds, but a tape drive must physically wind tape between reels
to read any one particular piece of data. As a result, tape drives have very slow average seek times.
However, the storage capacity of magnetic tapes is considerably more than other secondary storage
mediums.
\section{INPUT AND OUTPUT PERIPHERALS}
Input and output devices are typically housed externally to the main computer chassis. The following are either
standard or very common to many computer systems.
\subsection{INPUT DEVICES}
\paragraph{ KEYBOARDS}
 A keyboard is a device to input text and characters by depressing buttons. It is a typewriterstyle
device, which uses an arrangement of buttons or keys, to act as mechanical levers or electronic
switches. While most keyboard keys produce letters, numbers or signs (characters), other keys or
simultaneous key presses can produce actions or execute computer commands.
\paragraph{ MOUSE}
 A mouse is a pointing device that functions by detecting two-dimensional motion relative to its
supporting surface. Physically, a mouse consists of an object held under one of the user's hands, with
one or more buttons. The mouse's motion typically translates into the motion of a pointer on a display,
which allows for fine control of a graphical user interface. 
\paragraph{TRACKBALL}
 A trackball is a pointing device consisting of a ball held by a socket containing sensors to
detect a rotation of the ball about two axes—like an upside-down mouse with an exposed protruding
ball. The user rolls the ball with the thumb, fingers, or the palm of the hand to move a pointer.
\paragraph {TOUCHSCREEN}
 A touchscreen is an electronic visual display that the user can control through simple
or multi-touch gestures by touching the screen with one or more fingers. Some touchscreens can also
detect objects such as a stylus or ordinary or specially coated gloves. The user can use the touchscreen
to react to what is displayed and to control how it is displayed (for example by zooming the text size).
The touchscreen enables the user to interact directly with what is displayed, rather than using
a mouse, touchpad, or any other intermediate device. 
\paragraph {JOYSTICK}
 A joystick is an input device consisting of a stick that pivots on a base and reports its angle or
direction to the device it is controlling. Joysticks are often used to control video games, and usually
have one or more push-buttons whose state can also be read by the computer.
\paragraph {IMAGE SCANNER}
 In computing, an image scanner—often abbreviated to just scanner—is a device that
optically scans images, printed text, handwriting, or an object, and converts it to a digital image.
Common examples found in offices are variations of the desktop (or flatbed) scanner where the
document is placed on a glass window for scanning. Hand-held scanners, where the device is moved by
hand, have evolved from text scanning "wands" to 3D scanners used for industrial design, reverse
engineering, test and measurement, orthotics, gaming and other applications. Mechanically driven
scanners that move the document are typically used for large-format documents, where a flatbed design
would be impractical. Modern scanners typically use a charge-coupled device (CCD) or a Contact
Image Sensor (CIS) as the image sensor, whereas older drum scanners use a photomultiplier tube as the
image sensor. A rotary scanner, used for high-speed document scanning, is another type of drum
scanner, using a CCD array instead of a photomultiplier. Other types of scanners are planetary
scanners, which take photographs of books and documents, and 3D scanners, for producing threedimensional
models of objects.
\paragraph {MICROPHONE} 
A microphone is an acoustic to electric transducer or sensor that converts sound into an
electrical signal. Microphones are used to input audio data into the computer for processing.
8
\subsection{ OUTPUT DEVICES}
\paragraph{ PRINTERS}
 In computing, a printer is a peripheral which produces a representation of an electronic
document on physical media such as paper or transparency film. Many printers are local peripherals
connected directly to a nearby personal computer. Individual printers are often designed to support both
local and network connected users at the same time. Depending on the technology used, there can be
several variants of printers such as inkjet printers, laser printers, dot matrix printers, thermal printers
etc.
\paragraph{ COMPUTER MONITORS}
 A monitor or a display is an electronic visual display for computers. The monitor
comprises the display device, circuitry and an enclosure. The display device in modern monitors is
typically a thin film transistor liquid crystal display (TFT-LCD) thin panel, while older monitors use
a cathode ray tube (CRT) about as deep as the screen size.
\paragraph{ SPEAKERS}
 Computer speakers, or multimedia speakers, are speakers external to a computer, that disable
the lower fidelity built-in speaker. They often have a low-power internal amplifier. The standard audio
connection is a 3.5 mm (approximately 1/8 inch) stereo phone connector often color-coded lime green
(following the PC 99 standard) for computer sound cards. Analog A/V connectors often use shielded cables to
inhibit radio frequency interference (RFI) and noise. Some commonly used connectors are as follows:
 \paragraph{RCA}
  An RCA connector, sometimes called a phono connector or cinch connector, is a type of electrical
connector commonly used to carry audio and video signals. The connectors are also sometimes casually
referred to as A/V jacks.
\paragraph{ VGA}
 A Video Graphics Array (VGA) connector is a three-row 15-pin DE-15 connector which carries
video signals. The 15-pin VGA connector is found on many video cards, computer monitors, and high
definition television sets.
\paragraph{ DVI}
 Digital Visual Interface (DVI) is a video display interface developed by the Digital Display Working
Group (DDWG). The digital interface is used to connect a video source to a display device, such as
a computer monitor.
\paragraph{ HDMI}
 HDMI (High-Definition Multimedia Interface) is a compact audio/video interface for
transferring uncompressed video data and compressed/uncompressed digital audio data from a HDMIcompliant
device ("the source device") to a compatible computer monitor, video projector, digital
television, or digital audio device.
\end{document}

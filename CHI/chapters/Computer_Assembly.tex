


%\pagestyle{fancy}
\begin	{document}
\renewcommand{\abstractname}{\Large AIM}
\chapter{AIM} \vspace{0.5cm}
\\To study the assembly of a Computer from its components \vspace{0.5cm}
\paragraph{PROCEDURE\\}
Following are the steps involved in assembling a computer from its components:

{Step 1}: Open the case by removing the side panels.\\

{Step 2}: Install the Power Supply.\\

Power supply installation steps include the following:\\

\\Insert the power supply into the case.\\
\\Align the holes in the power supply with the holes in the case.\\
\\Secure the power supply to the case using the proper screws.\\

{Step 3}: Attach Components to the Motherboard\\
\\i) CPU on Motherboard\\
\\ The CPU and motherboard are sensitive to electrostatic discharge.
\\The CPU is secured to the socket on the motherboard with a locking assembly.
\\ Apply thermal compound which helps keep the CPU cool.\\
\\ii) Heat Sink/Fan Assembly\\
\\The Heat Sink/Fan Assembly is a two-part cooling device.
\\ The heat sink draws heat away from the CPU.\\
\\iii) Install RAM\\
\\RAM provides temporary data storage for the CPU and should be installed in the motherboard before the motherboard is placed in the computer case.
\\iv) Install motherboard in the computer case.\\
\\{Step 4}: Install the Hard Disk drive in the 3.5 inch internal drive bay using screws.\\
\\{Step 5}: Install optical disc drive and floppy disc drive in the external drive bays provided for the same.\\
\\{Step 6}: Install Adapter cards: Expansion cards such as NIC cards, video cards, sound cards etc. should be installed in the slots (PCI, PCIe) provided for the same in the motherboard.\\
\\{Step 7}: Connect all the internal power and data cables. Power cables distribute electricity from the power supply to the motherboard and other components. Data cables transmit data between the motherboard and storage devices, such as hard drives.\\
\\{Step 8}: Close the case by reattaching the side panels. Connect the external cables.\\
\\{Step 9}: Boot computer for first time. The BIOS will perform a Power On Self Test (POST) to check all of the internal components. If a device is malfunctioning, an error or beep code will be generated.\\
\\{Step 10}: If the computer is functioning properly, install a suitable operating system.

\end{document}

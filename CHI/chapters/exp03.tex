
\section*{AIM:}
To find the base address of the COM port in a system.
\section*{THEORY:}
COM is the original, yet still common, name of the serial port interface on IBM-PC compatible computers. It might refer not only to physical ports, but also to virtual ports, such as ports created by Bluetooth or USB-to-serial adapters. COM ports are usually associated with 4 memory addresses as shown below:

\begin{center}
\bgroup
\def\arraystretch{1.5}
\begin{tabular}{ |c|c|c| }
\hline
\textbf{NAME} & \textbf{ADDRESS} & \textbf{IRQ}\\
\hline
COM1 & 0x3f8 & 4\\
\hline
COM2 & 0x2f8 & 3\\
\hline
COM3 & 0x3e8 & 4\\
\hline
COM4 & 0x2e8 & 3\\
\hline
\end{tabular}
\egroup
\end{center}

The base addresses for the COM ports can be read from the BIOS data area. The addresses in the BIOS data
area that store the COM port base addresses are as follows:

\begin{center}
\bgroup
\def\arraystretch{1.5}
\begin{tabular}{ |c|c| }
\hline
\textbf{Start Address} & \textbf{Function}\\
\hline
0000:0400 & COM1's Base Address\\
\hline
0000:0402 & COM2's Base Adress\\
\hline
0000:0403 & COM3's Base Address\\
\hline
0000:0404 & COM4's Base Address\\
\hline
\end{tabular}
\egroup
\end{center}

\section*{PROCEDURE:}
Execute the following program to find the base address of the COM port.

\section*{PROGRAM:}
\begin{lstlisting}
#include <stdio.h>
#include <string.h>

int main()
{
     char s[100],string[100];
     int com=0;
     
     //--Executing BASH script to read all messages in kernel ring buffer,
     //  search for the string "ttyS" in these messages and
     //  redirect the search results into a text file for further use
     
     system("dmesg | grep ttyS > out.txt");
     
     FILE *fp;
     fp=fopen("out.txt","r");
     
     while( !feof(fp) )
     {
          fscanf(fp, "%s", string);
          if( strcmp(string,"I/O") == 0 )
          {
               fscanf(fp, "%s", s);
               printf("COM PORT%d ADDRESS IS %s\n",com,s);
          }
     }
     
     fclose(fp);
     return 0;
}
\end{lstlisting}

\section*{RESULT}
The program has been executed and the output, verified.
%
%

\chapter*{ Largest number in an array}
%
%
%
%
\section*{AIM:}
To find the largest number in an array of 10 numbers.

\section*{ALGORITHM:}
\textbf{\underline{STEP 1}:}Start

\textbf{\underline{STEP 2}:} Get first element to R1 register

\textbf{\underline{STEP 3}:}  Get the next element in A register

\textbf{\underline{STEP 4}:} Subtract the R1 register from the A register.

\textbf{\underline{STEP 5}:}  Check the carry flag.

\textbf{\underline{STEP 6}:} If carry is not there, then A register is greater that means 2nd data is larger than
1stData. So exchange the A and B register.

\textbf{\underline{STEP 7}:}  If carry is there, then B register is greater that means 1st data is larger than 2
nd Data. So do not exchange the A and B register.

\textbf{\underline{STEP 8}:}  Check the R0 register. If R0 is not equal to zero, repeat from the 2
nd step else Move the R1 register to a memory location.

\textbf{\underline{STEP 11}:} Stop


\section*{PROGRAM:}

\begin{lstlisting}
MOV R0,#10
MOV DPTR,#4200H
MOVX A,@DPTR
MOV R1,A
INC DPTR
DEC R0
LOOP1 :MOVX A,@DPTR
MOV R2,A
SUBB A,R1
JC NO
MOV R1,R2
NO :INC DPTR
DJNZ R0,LOOP1
MOV DPTR,#4300H
MOV A,R1
MOVX @DPTR,A
HLT :SJMP HLT
\end{lstlisting}

\section*{Sample Input}
(4200): 07 05 09 03 01 02 0A 04 06

\section*{Sample Output}
(4300): 0A

\section*{RESULT:}
The program has been executed and the output verified
%
%
%
%
